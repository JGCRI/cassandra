\documentclass[11pt]{article}
\usepackage{amsmath}

\title{{GCAM} Automation System Users' Guide}

\begin{document}

\maketitle

\section{Overview}
The GCAM automation system is intended to provide a way to run all the
steps of a GCAM analysis using a single command.  Such an analysis
might include several data preprocessing steps, a run of the GCAM core
model, and several data postprocessing steps.  The steps (called
``modules'' throughout this documentation) to be performed are laid
out in a configuration file, the format of which is designed to be
easily edited by hand or by an automated process (such as a graphical
front end).  The system deduces dependencies between modules based on
patterns of data use, and modules are run concurrently where possible.


\section{Installation and Setup}
\subsection{System Requirements}

The GCAM automation system has been developed and tested on Linux and
on OS~X.  The code is written portably in Python, so any version of
those operating systems from the past few years should work.  The code
requires Python~2.7.  Due to backward incompatibilities in Python~3.X,
later versions of the Python interpreter will \emph{not} work.

Output in NetCDF format requires the NetCDF-3 libraries.  These
libraries are available from Unidata (\texttt{unidata.ucar.edu}).

The modules that run GCAM analyses are distributed separately in their
own repositories.  In addition to run them you will need to install
the code for all of the modules you wish to run.  The FY-15 demos use
the GCAM core model and the hydrology and water downscaling code.
Both of these are distributed by PNNL.  The GCAM core model is a
stand-alone program, while the hydrology and water downscaling code
requires a copy of Matlab.

\subsection{How to install}

The GCAM automation system is hosted on GitHub and can be downloaded
from:\\
\texttt{https://github.com/JGCRI/gcam-driver}\\
The release used for FY15 runs for the Foresight project is
version~0.2.  To install the code download this release and unpack it
in the directory of your choice.  If you wish to produce netCDF
outputs, make sure the netCDF-3 libraries are installed, and set the
\texttt{NETCDF_INCDIR} environment variable to the location of the
netCDF include files, and the \texttt{NETCDF_LIBDIR} to the location
of the netCDF libraries.  Then, from the \texttt{src/C} subdirectory
of the release, run \texttt{make mat2nc} to build the netCDF
converter.

\section{How to run}

In order to run a GCAM calculation using the automation system, you
must first prepare a configuration file describing the GCAM modules to
include.  The configuration file is divided into sections, with each
section corresponding to a module.  The section begins with the name
of the module enclosed in square brackets and consists of a series of
key-value pairs separated by an $=$ sign.  Each module has its own set
of parameter options, which are explained below.  An example configuration
file is included with the code in the \texttt{example.cfg} file.

\subsection{The Global Module}
The Global module provides parameter values that will be used by multiple
other modules.  Most of these values are related to the locations of
files used to access GCAM data.  The name of this module is
\texttt{[Global]}.
\begin{description}
\item[ModelInterface] The location of the \texttt{ModelInterface.jar}
  Java file.  This will have been installed along with the GCAM core
  model.
\item[DBXMLlib] The location of the DBXML libraries.  These were also
  installed with the GCAM core model.
\item[inputdir] The location of the gcam-driver input data.  For now,
  this is always \texttt{./input-data}.  It is included as a settable
  parameter to allow for future expansion.
\item[rgnconfig] The directory containing the files that describe how
  the world is divided up into regions.  This directory path is
  relative to the directory specified for \texttt{inputdir}.  There
  are currently three to choose from:  \texttt{rgn14}, \texttt{rgn32},
  and \texttt{rgnchn}.
\end{description}

\subsection{The Hydrology Module}
The Hydrology module runs the hydrology for the future scenario and
makes the results available to other modules.  The name of the module
is \texttt{[HydroModule]}
\begin{description}
\item[workdir] The working directory for the hydrology calculation.
  Generally, this is the location where the gcam-hydro code is
  installed.
\item[inputdir] Location of the CMIP5 model output used as input by the
  hydrology calculation.
\item[outputdir] Directory into which to write the module's output
  data.
\item[init-storage-file] File giving the initial condition for the
  water stored in the soil column.  This file is included with the
  gcam-hydro distribution, but it can be replaced if something better
  is available.
\item[logfile] File into which to write screen output from the matlab
  code.  This is mostly intended for debugging problems with the
  matlab calculation.
\item[gcm] The name of the earth system model to run the hydrology
  on.  This will be used to construct the name of both the input and
  output files.
\item[scenario] The RCP scenario to use for the calculation.  This
  will also be a component in the input and output file names.
\item[runid] The ensemble ID and start and end months for the input
  CMIP5 data.  This will be used to identify the correct input files
  in the CMIP5 data directory.
\item[clobber] Either ``True'' or ``False''.  If False, then the code
  will check to see whether the anticipated results already exist.  If
  the do, then the hydrology calculation will not run; the preexisting
  results will be used.  If True, then the hydrology calculation will
  overwrite the preexisting results, if any.
\end{description}

\subsection{The Historical Hydrology Module}

Once the configuration file has been prepared, the program can be run
from the command line by changing to the 
top-level gcam-driver directory and running:\\
\texttt{./gcam-driver <configfile>}\\ 


\appendix
\section{Technical Reference}
Technical details for the python modules that make up the GCAM
automation system can be obtained by using the \texttt{pydoc} command
included with most python distributions.  For this to work the
PYTHONPATH environment variable must include the \texttt{src}
directory of the \texttt{gcam-driver} installation.  For csh-type
command shells this can be accomplished by running:\\ 
\texttt{setenv PYTHONPATH /path/to/gcam-driver/src}\\
For bourne-type shells the command is:\\
\texttt{export PYTHONPATH=/path/to/gcam-driver/src}.

The information given in the pydoc documentation pertains mostly to
the internals of the GCAM automation system, including function
arguments and return values and descriptions of internal data
structures.  It will therefore be useful primarily to users interested
in extending the system.

\end{document}